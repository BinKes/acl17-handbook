Obsessive-compulsive disorder (OCD) is an anxiety-based disorder that affects around 2.5\% of the population. A common treatment for OCD is exposure therapy, where the patient repeatedly confronts a feared experience, which has the long-term effect of decreasing their anxiety. Some exposures consist of reading and writing stories about an imagined anxiety-provoking scenario. In this paper, we present a technology that enables patients to interactively contribute to exposure stories by supplying natural language input (typed or spoken) that advances a scenario. This interactivity could potentially increase the patient's sense of immersion in an exposure and contribute to its success. We introduce the NLP task behind processing inputs to predict new events in the scenario, and describe our initial approach. We then illustrate the future possibility of this work with an example of an exposure scenario authored with our application.
