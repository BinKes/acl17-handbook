People typically assume that killers are mentally ill or fundamentally different from the rest of humanity. Similarly, people often associate mental health conditions (such as schizophrenia or autism) with violence and otherness - treatable perhaps, but not empathically understandable. We take a dictionary approach to explore word use in a set of autobiographies, comparing the narratives of 2 killers (Adolf Hitler and Elliot Rodger) and 39 non-killers. Although results suggest several dimensions that differentiate these autobiographies - such as sentiment, temporal orientation, and references to death - they appear to reflect subject matter rather than psychology per se. Additionally, the Rodger text shows roughly typical developmental arcs in its use of words relating to friends, family, sex, and affect. From these data, we discuss the challenges of understanding killers and people in general.
