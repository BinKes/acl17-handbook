Recent development of large-scale question answering (QA) datasets triggered a substantial amount of research into end-to-end neural architectures for QA. Increasingly complex systems have been conceived without comparison to simpler neural baseline systems that would justify their complexity. In this work, we propose a simple heuristic that guides the development of neural baseline systems for the extractive QA task. We find that there are two ingredients necessary for building a high-performing neural QA system: first, the awareness of question words while processing the context and second, a composition function that goes beyond simple bag-of-words modeling, such as recurrent neural networks. Our results show that FastQA, a system that meets these two requirements, can achieve very competitive performance compared with existing models. We argue that this surprising finding puts results of previous systems and the complexity of recent QA datasets into perspective.
