Recent work has explored the syntactic abilities of RNNs using the subject-verb agreement task, which diagnoses sensitivity to sentence structure. RNNs performed this task well in common cases, but faltered in complex sentences (Linzen et al., 2016). We test whether these errors are due to inherent limitations of the architecture or to the relatively indirect supervision provided by most agreement dependencies in a corpus. We trained a single RNN to perform both the agreement task and an additional task, either CCG supertagging or language modeling. Multi-task training led to significantly lower error rates, in particular on complex sentences, suggesting that RNNs have the ability to evolve more sophisticated syn- tactic representations than shown before. We also show that easily available agreement training data can improve performance on other syntactic tasks, in particular when only a limited amount of training data is available for those tasks. The multi-task paradigm can also be leveraged to inject grammatical knowledge into language models.
