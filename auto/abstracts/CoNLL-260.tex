This paper explores a divisive hierarchical clustering algorithm based on the well-known Obligatory Contour Principle in phonology.  The purpose is twofold: to see if such an algorithm could be used for unsupervised classification of phonemes or graphemes in corpora, and to investigate whether this purported universal constraint really holds for several classes of phonological distinctive features.  The algorithm achieves very high accuracies in an unsupervised setting of inferring a consonant-vowel distinction, and also has a strong tendency to detect coronal phonemes in an unsupervised fashion. Remaining classes, however, do not correspond as neatly to phonological distinctive feature splits.  While the results offer only mixed support for a universal Obligatory Contour Principle, the algorithm can be very useful for many NLP tasks due to the high accuracy in revealing consonant/vowel/coronal distinctions.
