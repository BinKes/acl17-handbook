I explore the hypothesis that conventional neural network models (e.g., recurrent neural networks) are incorrectly biased for making linguistically sensible generalizations when learning, and that a better class of models is based on architectures that reflect hierarchical structures for which considerable behavioral evidence exists. I focus on the problem of modeling and representing the meanings of sentences. On the generation front, I introduce recurrent neural network grammars (RNNGs), a joint, generative model of phrase-structure trees and sentences. RNNGs operate via a recursive syntactic process reminiscent of probabilistic context-free grammar generation, but decisions are parameterized using RNNs that condition on the entire (top-down, left-to-right) syntactic derivation history, thus relaxing context-free independence assumptions, while retaining a bias toward explaining decisions via ``syntactically local'' conditioning contexts. Experiments show that RNNGs obtain better results in generating language than models that don't exploit linguistic structure. On the representation front, I explore unsupervised learning of syntactic structures based on distant semantic supervision using a reinforcement-learning algorithm. The learner seeks a syntactic structure that provides a compositional architecture that produces a good representation for a downstream semantic task. Although the inferred structures are quite different from traditional syntactic analyses, the performance on the downstream tasks surpasses that of systems that use sequential RNNs and tree-structured RNNs based on treebank dependencies. This is joint work with Adhi Kuncoro, Dani Yogatama, Miguel Ballesteros, Phil Blunsom, Ed Grefenstette, Wang Ling, and Noah A. Smith.
