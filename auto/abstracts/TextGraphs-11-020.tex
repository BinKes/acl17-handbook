This paper introduces a new, graph-based view of the data of the FrameNet project, which we hope will make it easier to understand the mixture of semantic and syntactic information contained in FrameNet annotation.  We show how English FrameNet and other Frame Semantic resources can be represented as sets of interconnected graphs of frames, frame elements, semantic types, and annotated instances of them in text.  We display examples of the new graphical representation based on the annotations, which combine Frame Semantics and Construction Grammar, thus capturing most of the syntax and semantics of each sentence.  We consider how graph theory could help researchers to make better use of FrameNet data for tasks such as automatic Frame Semantic role labeling, paraphrasing, and translation.              Finally, we describe the development of FrameNet-like lexical resources for other languages in the current Multilingual FrameNet project.  which seeks to discover cross-lingual alignments, both in the lexicon (for frames and lexical units within frames) and across parallel or comparable texts.  We conclude with an example showing graphically the semantic and syntactic similarities and differences between parallel sentences in English and Japanese.  We will release software for displaying such graphs from the current data releases.
