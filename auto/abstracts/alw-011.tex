In this paper, we use a new categorical form of multidimensional register analysis to identify the main dimensions of functional linguistic variation in a corpus of abusive language, consisting of racist and sexist Tweets. By analysing the use of a wide variety of parts-of-speech and grammatical constructions, as well as various features related to Twitter and computer-mediated communication, we discover three dimensions of linguistic variation in this corpus, which we interpret as being related to the degree of interactive, antagonistic and attitudinal language exhibited by individual Tweets. We then demonstrate that there is a significant functional difference between racist and sexist Tweets, with sexists Tweets tending to be more interactive and attitudinal than racist Tweets.
