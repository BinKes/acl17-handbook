As the body of research on abusive language detection and analysis grows, there is a need for critical consideration of the relationships between different subtasks that have been grouped under this label. Based on work on hate speech, cyberbullying, and online abuse we propose a typology that captures central similarities and differences between subtasks and discuss the implications of this for data annotation and feature construction. We emphasize the practical actions that can be taken by researchers to best approach their abusive language detection subtask of interest.
