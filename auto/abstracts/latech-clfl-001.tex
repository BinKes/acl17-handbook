The post-modern novel ``Wittgenstein's Mistress'' by David Markson (1988) presents the reader with a very challenging non-linear narrative, that itself appears to one of the novel's themes. We present a distant reading of this work designed to complement a close reading of it by David Foster Wallace (1990).   Using a combination of text analysis, entity recognition and networks, we plot repetitive structures in the novel's narrative relating them to its critical analysis.
