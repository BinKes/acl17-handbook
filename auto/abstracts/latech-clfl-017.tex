We present a data-driven approach to investigate intra-textual variation by combining entropy and surprisal. With this approach we detect linguistic variation based on phrasal lexico-grammatical patterns across sections of research articles. Entropy is used to detect patterns typical of specific sections. Surprisal is used to differentiate between more and less informationally-loaded patterns as well as type of information (topical vs. stylistic). While we here focus on research articles in biology/genetics, the methodology is especially interesting for digital humanities scholars, as it can be applied to any text type or domain and combined with additional variables (e.g. time, author or social group).
