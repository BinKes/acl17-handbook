In this paper, we evaluate the predictability of tweets associated with controversial versus non-controversial topics. As a first step, we crowd-sourced the scoring of a predefined set of topics on a Likert scale from non-controversial to controversial. Our feature set entails and goes beyond sentiment features, e.g., by leveraging empathic language and other features that have been previously used but are new for this particular study. We find focusing on the structural characteristics of tweets to be beneficial for this task. Using a combination of emphatic, language-specific, and Twitter-specific features for supervised learning resulted in 87\% accuracy (F1) for cross-validation of the training set and 63.4\% accuracy when using the test set. Our analysis shows that features specific to Twitter or social media, in general, are more prevalent in tweets on controversial topics than in non-controversial ones. To test the premise of the paper, we conducted two additional sets of experiments, which led to mixed results. This finding will inform our future investigations into the relationship between language use on social media and the perceived controversiality of topics.
