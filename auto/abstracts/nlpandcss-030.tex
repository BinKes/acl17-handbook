Personality plays a decisive role in how people behave in different scenarios, including online social media. Researchers have used such data to study how personality can be predicted from language use. In this paper, we study phrase choice as a particular stylistic linguistic difference, as opposed to the mostly topical differences identified previously. Building on previous work on demographic preferences, we quantify differences in paraphrase choice from a massive Facebook data set with posts from over 115,000 users. We quantify the predictive power of phrase choice in user profiling and use phrase choice to study psycholinguistic hypotheses. This work is relevant to future applications that aim to personalize text generation to specific personality types.
