Extracting time expressions from free text is a fundamental task for many applications. We analyze the time expressions from four datasets and find that only a small group of words are used to express time information, and the words in time expressions demonstrate similar syntactic behaviour. Based on the findings, we propose a type-based approach, named SynTime, to recognize time expressions. Specifically, we define three main syntactic token types, namely time token, modifier, and numeral, to group time-related regular expressions over tokens. On the types we design general heuristic rules to recognize time expressions. In recognition, SynTime first identifies the time tokens from raw text, then searches their surroundings for modifiers and numerals to form time segments, and finally merges the time segments to time expressions. As a light-weight rule-based tagger, SynTime runs in real time, and can be easily expanded by simply adding keywords for the text of different types and of different domains. Experiment on benchmark datasets and tweets data shows that SynTime outperforms state-of-the-art methods.
