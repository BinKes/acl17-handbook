Sarcasm is a form of speech in which speakers say the opposite of what they truly mean in order to convey a strong sentiment. In other words, ''Sarcasm is the giant chasm between what I say, and the person who doesn't get it.''. In this paper we present the novel task of sarcasm interpretation, defined as the generation of a non-sarcastic utterance conveying the same message as the original sarcastic one. We introduce a novel dataset of 3000 sarcastic tweets, each interpreted by five human judges. Addressing the task as monolingual machine translation (MT), we experiment with MT algorithms and evaluation measures. We then present SIGN: an MT based sarcasm interpretation algorithm that targets sentiment words, a defining element of textual sarcasm. We show that while the scores of n-gram based automatic measures are similar for all interpretation models, SIGN's interpretations are scored higher by humans for adequacy and sentiment polarity. We conclude with a discussion on future research directions for our new task.
