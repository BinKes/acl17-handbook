Translation has played an important role in trade, law, commerce, politics, and literature for thousands of years. Translators have always tried to be invisible; ideal translations should look as if they were written originally in the target language. We show that traces of the source language remain in the translation product to the extent that it is possible to uncover the history of the source language by looking only at the translation. Specifically, we automatically reconstruct phylogenetic language trees from monolingual texts (translated from several source languages). The signal of the source language is so powerful that it is retained even after two phases of translation. This strongly indicates that source language interference is the most dominant characteristic of translated texts, overshadowing the more subtle signals of universal properties of translation.
