It has been shown that Chinese poems can be successfully generated by sequence-to-sequence neural models, particularly with the attention mechanism. A potential problem of this approach, however, is that neural models can only learn abstract rules, while poem generation is a highly creative process that involves not only rules but also innovations for which pure statistical models are not appropriate in principle. This work proposes a memory augmented neural model for Chinese poem generation, where the neural model and the augmented memory work together to balance the requirements of linguistic accordance and aesthetic innovation, leading to innovative generations that are still rule-compliant. In addition, it is found that the memory mechanism provides interesting flexibility that can be used to generate poems with different styles.
