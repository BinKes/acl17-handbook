Automated processing of historical texts often relies on pre-normalization to modern word forms. Training encoder-decoder architectures to solve such problems typically requires a lot of training data, which is not available for the named task. We address this problem by using several novel encoder-decoder architectures, including a multi-task learning (MTL) architecture using a grapheme-to-phoneme dictionary as auxiliary data, pushing the state-of-the-art by an absolute 2\% increase in performance. We analyze the induced models across 44 different texts from Early New High German. Interestingly, we observe that, as previously conjectured, multi-task learning can learn to focus attention during decoding, in ways remarkably similar to recently proposed attention mechanisms. This, we believe, is an important step toward understanding how MTL works.
