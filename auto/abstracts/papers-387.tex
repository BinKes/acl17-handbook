Cognitive NLP systems- i.e., NLP systems that make use of behavioral data - augment traditional text-based features with cognitive features extracted from eye-movement patterns, EEG signals, brain-imaging etc. Such extraction of features is typically manual. We contend that manual extraction of features may not be the best way to tackle text subtleties that characteristically prevail in complex classification tasks like Sentiment Analysis and Sarcasm Detection, and that even the extraction and choice of features should be delegated to the learning system.  We introduce a framework to automatically extract cognitive features from the eye-movement/gaze data of human readers reading the text and use them as features along with textual features for the tasks of sentiment polarity and sarcasm detection. Our proposed framework is based on Convolutional Neural Network (CNN). The CNN learns features from both gaze and text and uses them to classify the input text. We test our technique on published sentiment and sarcasm labeled datasets, enriched with gaze information, to show that using a combination of automatically learned text and gaze features often yields better classification performance over (i)  CNN based systems that rely on text input alone and (ii) existing systems that rely on handcrafted gaze and textual features.
