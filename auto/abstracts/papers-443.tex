The referring expressions (REs) produced by a natural language generation (NLG) system can be misunderstood by the hearer, even when they are semantically correct. In an interactive setting, the NLG system can try to recognize such misunderstandings and correct them. We present an algorithm for generating corrective REs that use contrastive focus (``no, the BLUE button'') to emphasize the information the hearer most likely misunderstood. We show empirically that these contrastive REs are preferred over REs without contrast marking.
