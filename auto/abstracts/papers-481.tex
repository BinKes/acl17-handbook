In this paper, we aim to understand whether current language and vision (LaVi) models truly grasp the interaction between the two modalities. To this end, we propose an extension of the MS-COCO dataset, FOIL-COCO, which associates images with both correct and `foil' captions, that is, descriptions of the image that are highly similar to the original ones, but contain one single mistake (`foil word'). We show that current LaVi models fall into the traps of this data and perform badly on three tasks: a) caption  classification (correct vs. foil); b) foil word detection; c) foil word correction. Humans, in contrast, have near-perfect performance on those tasks. We demonstrate that merely utilising language cues is not enough to model FOIL-COCO and that it challenges the state-of-the-art by requiring a fine-grained understanding of the relation between text and image.
