Morphologically rich languages accentuate two properties of distributional vector space models: 1) the difficulty of inducing accurate representations for low-frequency word forms; and 2) insensitivity to distinct lexical relations that have similar distributional signatures. These effects are detrimental for language understanding systems, which may infer that  'inexpensive' is a rephrasing for 'expensive' or may not associate 'acquire' with 'acquires'. In this work, we propose a novel morph-fitting procedure which moves past the use of curated semantic lexicons for improving distributional vector spaces. Instead, our method injects morphological constraints generated using simple language-specific rules, pulling inflectional forms of the same word close together and pushing derivational antonyms far apart. In intrinsic evaluation over four languages, we show that our approach: 1) improves low-frequency word estimates; and 2) boosts the semantic quality of the entire word vector collection. Finally, we show that morph-fitted vectors yield large gains in the downstream task of dialogue state tracking, highlighting the importance of morphology for tackling long-tail phenomena in language understanding tasks.
