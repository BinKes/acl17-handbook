The sociolinguistic construct of stancetaking describes the activities through which discourse participants create and signal relationships to their interlocutors, to the topic of discussion, and to the talk itself. Stancetaking underlies a wide range of interactional phenomena, relating to formality, politeness, affect, and subjectivity. We present a computational approach to stancetaking, in which we build a theoretically-motivated lexicon of stance markers, and then use multidimensional analysis to identify a set of underlying stance dimensions. We validate these dimensions intrinscially and extrinsically, showing that they are internally coherent, match pre-registered hypotheses, and correlate with social phenomena.
