We propose two novel methodologies for the automatic generation of rhythmic poetry in a variety of forms. The first approach uses a neural language model trained on a phonetic encoding to learn an implicit representation of both the form and content of English poetry. This model can effectively learn common poetic devices such as rhyme, rhythm and alliteration. The second approach considers poetry generation as a constraint satisfaction problem where a generative neural language model is tasked with learning a representation of content, and a discriminative weighted finite state machine constrains it on the basis of form. By manipulating the constraints of the latter model, we can generate coherent poetry with arbitrary forms and themes. A large-scale extrinsic evaluation demonstrated that participants consider machine-generated poems to be written by humans 54\% of the time. In addition, participants rated a machine-generated poem to be the best amongst all evaluated.
