Cultural fit is widely believed to affect the success of individuals and the groups to which they belong. Yet it remains an elusive, poorly measured construct. Recent research draws on computational linguistics to measure cultural fit but overlooks asymmetries in cultural adaptation. By contrast, we develop a directed, dynamic measure of cultural fit based on linguistic alignment, which estimates the influence of one person's word use on another's and distinguishes between two enculturation mechanisms: internalization and self-regulation. We use this measure to trace employees' enculturation trajectories over a large, multi-year corpus of corporate emails and find that patterns of alignment in the first six months of employment are predictive of individuals' downstream outcomes, especially involuntary exit. Further predictive analyses suggest referential alignment plays an overlooked role in linguistic alignment.
