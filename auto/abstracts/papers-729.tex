How fake news goes viral via social media? How does its propagation pattern differ from real stories? In this paper, we attempt to address the problem of identifying rumors, i.e., fake information, out of microblog posts based on their propagation structure. We firstly model microblog posts diffusion with propagation trees, which provide valuable clues on how an original message is transmitted and developed over time. We then propose a kernel-based method called Propagation Tree Kernel, which captures high-order patterns differentiating different types of rumors by evaluating the similarities between their propagation tree structures. Experimental results on two real-world datasets demonstrate that the proposed kernel-based approach can detect rumors more quickly and accurately than state-of-the-art rumor detection models.
