Linguistic typology studies the range of structures present in human language. The main goal of the field is to discover which sets of possible phenomena are universal, and which are merely frequent. For example, all languages have vowels, while most---but not all---languages have an /u/ sound. In this paper we present the first probabilistic treatment of a basic question in phonological typology: What makes a natural vowel inventory?  We introduce a series of deep stochastic point processes, and contrast them with previous computational, simulation-based approaches.  We provide a comprehensive suite of experiments on over 200 distinct languages.
