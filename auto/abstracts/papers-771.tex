An interesting aspect of structured prediction is the evaluation of an output structure against the gold standard. Especially in the loss-augmented setting, the need of finding the max-violating constraint has severely limited the expressivity of effective loss functions. In this paper, we trade off exact computation for enabling the use and study of more complex loss functions for coreference resolution. Most interestingly, we show that such functions can be (i) automatically learned also from controversial but commonly accepted coreference measures, e.g., MELA, and (ii) successfully used in learning algorithms. The accurate model comparison on the standard CoNLL-2012 setting shows the benefit of more expressive loss functions.
