We presents in this paper our approach for modeling inter-topic preferences of Twitter users: for example, ``those who agree with the Trans-Pacific Partnership (TPP) also agree with free trade''. This kind of knowledge is useful not only for stance detection across multiple topics but also for various real-world applications including public opinion survey, electoral prediction, electoral campaigns, and online debates. In order to extract users' preferences on Twitter, we design linguistic patterns in which people agree and disagree about specific topics (e.g., ``A is completely wrong''). By applying these linguistic patterns to a collection of tweets, we extract statements agreeing and disagreeing with various topics. Inspired by previous work on item recommendation, we formalize the task of modeling inter-topic preferences as matrix factorization: representing users' preference as a user-topic matrix and mapping both users and topics onto a latent feature space that abstracts the preferences. Our experimental results demonstrate both that our presented approach is useful in predicting missing preferences of users and that the latent vector representations of topics successfully encode inter-topic preferences.
