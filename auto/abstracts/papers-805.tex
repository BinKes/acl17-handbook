Text similarity measures are used in multiple tasks such as plagiarism detection, information ranking and recognition of paraphrases and textual entailment. While recent advances in deep learning highlighted the relevance of sequential models in natural language generation, existing similarity measures do not fully exploit the sequential nature of language. Examples of such similarity measures include n-grams and skip-grams overlap which rely on distinct slices of the input texts. In this paper we present a novel text similarity measure inspired from a common representation in DNA sequence alignment algorithms. The new measure, called TextFlow, represents input text pairs as continuous curves and uses both the actual position of the words and sequence matching to compute the similarity value. Our experiments on 8 different datasets show very encouraging results in paraphrase detection, textual entailment recognition and ranking relevance.
