Little attention has been paid to distributional compositional methods which employ syntactically structured vector models. As word vectors belonging to different syntactic categories                                      have incompatible syntactic distributions, no trivial compositional operation can be applied to combine them into a new compositional vector. In this article, we generalize the method described by Erk and Padó (2009) by proposing a dependency-base framework that contextualize not only lemmas but also selectional preferences.  The main contribution of the article is to expand their model to a fully compositional framework in which syntactic dependencies are put at the core of semantic composition. We claim that semantic composition is mainly driven by syntactic dependencies. Each syntactic dependency generates two new compositional vectors representing the contextualized sense of the two related lemmas.  The sequential application of the compositional operations associated to the dependencies results in as many contextualized vectors as lemmas the composite expression contains. At the end of the semantic process, we do not obtain a single compositional vector representing the semantic denotation of the whole composite expression, but one contextualized vector for each lemma of the whole expression. Our method avoids the troublesome high-order tensor representations by defining lemmas and selectional restrictions as first-order tensors (i.e. standard vectors). A corpus-based experiment is performed to both evaluate the quality of the compositional vectors built with our strategy, and to compare them to other approaches on distributional compositional semantics. The experiments show that our dependency-based compositional method performs as  (or even better than) the state-of-the-art.
