While neural machine translation (NMT) models provide improved translation quality in an elegant framework, it is less clear what they learn about language. Recent work has started evaluating the quality of vector representations learned by NMT models on morphological and semantic tasks. In this paper, we investigate the representations learned at different layers of NMT encoders. We train NMT systems on parallel data and use the models to extract features for training a classifier on two tasks: part-of-speech (POS) and semantic tagging. We then measure the performance of the classifier as a proxy to the quality of the original NMT model for the given task. Our quantitative analysis yields interesting insights regarding representation learning in NMT models. For instance, we find that higher layers are better at learning semantics while lower layers are better for POS tagging.
