The blind application of machine learning runs the risk of amplifying biases present in data. Such a danger is facing us with word embedding, a popular framework to represent text data as vectors which has been used in many machine learning and natural language processing tasks. We show that even word embeddings trained on Google News articles exhibit female/male gender stereotypes to a disturbing extent. This raises concerns because their widespread use, as we describe, often tends to amplify these biases. Geometrically, gender bias is first shown to be captured by a direction in the word embedding. Second, gender neutral words are shown to be linearly separable from gender definition words in the word embedding. Using these properties, we provide a methodology for modifying an embedding to remove gender stereotypes, such as the association between the words receptionist and female, while maintaining desired associations such as between the words queen and female. Using crowd-worker evaluation as well as standard benchmarks, we empirically demonstrate that our algorithms significantly reduce gender bias in embeddings while preserving the its useful properties such as the ability to cluster related concepts and to solve analogy tasks. The resulting embeddings can be used in applications without amplifying gender bias.
