We study the skip-thought model with neighborhood information as weak supervision. More specifically, we propose a skip-thought neighbor model to consider the adjacent sentences as a neighborhood. We train our skip-thought neighbor model on a large corpus with continuous sentences, and then evaluate the trained model on 7 tasks, which include semantic relatedness, paraphrase detection, and classification benchmarks. Both quantitative comparison and qualitative investigation are conducted. We empirically show that, our skip-thought neighbor model performs as well as the skip-thought model on evaluation tasks. In addition, we found that, incorporating an autoencoder path in our model didn't aid our model to perform better, while it hurts the performance of the skip-thought model.
