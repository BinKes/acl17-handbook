Robot-directed communication is variable, and may change based on human perception of robot capabilities. To collect training data for a dialogue system and to investigate possible communication changes over time, we developed a Wizard-of-Oz study that (a) simulates a robot's limited understanding, and (b) collects dialogues where human participants build a progressively better mental model of the robot's understanding. With ten participants, we collected ten hours of human-robot dialogue. We analyzed the structure of instructions that participants gave to a remote robot before it responded. Our findings show a general initial preference for including metric information (e.g., move forward 3 feet) over landmarks (e.g., move to the desk) in motion commands, but this decreased over time, suggesting changes in perception.
