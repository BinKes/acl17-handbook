While sentiment analysis in English has achieved significant progress, it remains a challenging task in Arabic given the rich morphology of the language. It becomes more challenging when applied to Twitter data that comes with additional sources of noise including dialects, misspellings, grammatical mistakes, code switching and the use of non-textual objects to express sentiments. This paper describes the ``OMAM'' systems that we developed as part of SemEval-2017 task 4. We evaluate English state-of-the-art methods on Arabic tweets for subtask A. As for the remaining subtasks, we introduce a topic-based approach that accounts for topic specificities by predicting topics or domains of upcoming tweets, and then using this information to predict their sentiment. Results indicate that applying the English state-of-the-art method to Arabic has achieved solid results without significant enhancements. Furthermore, the topic-based method ranked 1st in subtasks C and E, and 2nd in subtask D.
