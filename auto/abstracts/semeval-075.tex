This paper describes our system participating in the SemEval-2017 Task 7, for the subtasks of homographic pun location and homographic pun interpretation. For pun interpretation, we use a knowledge-based Word Sense Disambiguation (WSD) method based on sense embeddings. Pun-based jokes can be divided into two parts, each containing information about the two distinct senses of the pun. To exploit this structure we split the context that is input to the WSD system into two local contexts and find the best sense for each of them. We use the output of pun interpretation for pun location. As we expect the two meanings of a pun to be very dissimilar, we compute sense embedding cosine distances for each sense-pair and select the word that has the highest distance. We describe experiments on different methods of splitting the context and compare our method to several baselines. We find evidence supporting our hypotheses and obtain competitive results for pun interpretation.
