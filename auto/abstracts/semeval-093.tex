This paper describes the approach we used for SemEval-2017 Task 4: Sentiment Analysis in Twitter. Topic-based (target-dependent) sentiment analysis has become attractive and been used in some applications recently, but it is still a challenging research task. In our approach, we take the left and right context of a target into consideration when generating polarity classification features.  We use two types of word embeddings in our classifiers: the general word embeddings learned from 200 million tweets, and sentiment-specific word embeddings learned from 10 million tweets using distance supervision.  We also incorporate a text feature model in our algorithm. This model produces features based on text negation, tf.idf weighting scheme, and a Rocchio text classification method. We participated in four subtasks (B, C, D \& E for English), all of which are about topic-based message polarity classification. Our team is ranked \#6 in subtask B, \#3 by MAEu and \#9 by MAEm in subtask C, \#3 using RAE and \#6 using KLD in subtask D, and \#3 in subtask E.
