This paper describes the system devel-oped for SemEval 2017 task 6: \#HashTagWars -Learning a Sense of Hu-mor. Learning to recognize sense of hu-mor is the important task for language understanding applications. Different set of features based on frequency of words, structure of tweets and semantics are used in this system to identify the presence of humor in tweets. Supervised machine learning approaches, Multilayer percep-tron and Naïve Bayes are used to classify the tweets in to three level of sense of humor. For given Hashtag, the system finds the funniest tweet and predicts the amount of funniness of all the other tweets. In official submitted runs, we have achieved 0.506 accuracy using mul-tilayer perceptron in subtask-A and 0.938 distance in subtask-B. Using Naïve bayes in subtask-B, the system achieved 0.949 distance. Apart from official runs, this system have scored 0.751 accuracy in subtask-A using SVM. But still there is a wide room for improvement in system.
