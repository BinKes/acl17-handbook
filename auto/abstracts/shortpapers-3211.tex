Lexical features are a major source of information in state-of-the-art coreference resolvers. Lexical features implicitly model some of the linguistic phenomena at a fine granularity level. They are especially useful for representing the context of mentions. In this paper we investigate a drawback of using many lexical features in state-of-the-art coreference resolvers. We show that if coreference resolvers mainly rely on lexical features, they can hardly generalize to unseen domains. Furthermore, we show that the current coreference resolution evaluation is clearly flawed by only evaluating on a specific split of a specific dataset in which there is a notable overlap between the training, development and test sets.
