The availability of the Rhetorical Structure Theory (RST) Discourse Treebank has spurred substantial research into discourse analysis of written texts; however, limited research has been conducted to date on RST annotation and parsing of spoken language, in particular, non-native spontaneous speech. Considering that the measurement of discourse coherence is typically a key metric in human scoring rubrics for assessments of spoken language, we initiated a research effort to obtain RST annotations of a large number of non-native spoken responses from a standardized assessment of academic English proficiency. The resulting inter-annotator kappa agreements on the three different levels of Span, Nuclearity, and Relation are 0.848, 0.766, and 0.653, respectively. Furthermore, a set of features was explored to evaluate the discourse structure of non-native spontaneous speech based on these annotations; the highest performing feature resulted in a correlation of 0.612 with scores of discourse coherence provided by expert human raters.
