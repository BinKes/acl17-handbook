A common test administered during neurological examination is the semantic fluency test, in which the patient must list as many examples of a given semantic category as possible under timed conditions. Poor performance is associated with neurological conditions characterized by impairments in executive function, such as dementia, schizophrenia, and autism spectrum disorder (ASD). Methods for analyzing semantic fluency responses at the level of detail necessary to uncover these differences have typically relied on subjective manual annotation. In this paper, we explore automated approaches for scoring semantic fluency responses that leverage ontological resources and distributional semantic models to characterize the semantic fluency responses produced by young children with and without ASD. Using these methods, we find significant differences in the semantic fluency responses of children with ASD, demonstrating the utility of using objective methods for clinical language analysis. \end{abstract}
