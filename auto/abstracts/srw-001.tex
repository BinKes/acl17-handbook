Psychiatry is an area of medicine that strongly bases its diagnoses on the psychiatrist's subjective appreciation. More precisely, speech is used almost exclusively as a window to the patient's mind. Few other cues are available to objectively justify a diagnostic, unlike what happens in other disciplines which count on laboratory tests or imaging procedures, such as X-rays. Daily practice is based on the use of semi-structured interviews and standardized tests to build the diagnoses, heavily relying on her personal experience. This methodology has a big problem: diagnoses are commonly validated a posteriori in function of how the pharmacological treatment works. This validation cannot be done until months after the start of the treatment and, if the patient condition does not improve, the psychiatrist often changes the diagnosis and along with the pharmacological treatment. This delay prolongs the patient's suffering until the correct diagnosis is found. According to NIMH, more than 1\% and 2\% of US population is affected by Schizophrenia and Bipolar Disorder, respectively. Moreover, the WHO reported that the global cost of mental illness reached \$2.5T in 2010 [1] . The task of diagnosis, largely simplified, mainly consists of understanding the mind state through the extraction of patterns from the patient's speech and finding the best matching pathology in the standard diagnostic literature. This pipeline, consisting of extracting patterns and then classifying them, loosely resembles the common pipelines used in supervised learning schema. Therefore, we propose to augment the psychiatrists' diagnosis toolbox with an artificial intelligence system based on natural language processing and machine learning algorithms. The proposed system would assist in the diagnostic using a patient's speech as input. The understanding and insights obtained from customizing these systems to specific pathologies is likely to be more broadly applicable to other NLP tasks, therefore we expect to make contributions not only for psychiatry but also within the computer science community. We intend to develop these ideas and evaluate them beyond the lab setting. Our end goal is to make it possible for a practitioner to integrate our tools into her daily practice with minimal effort
