Inquiry is fundamental to communication, and machines cannot effectively collaborate with humans unless they can ask questions. In this thesis work, we explore how can we teach machines to ask clarification questions when faced with uncertainty, a goal of increasing importance in today's automated society. We do a preliminary study using data from StackExchange, a plentiful online resource where people routinely ask clarifying questions to posts so that they can better offer assistance to the original poster. We build neural network models inspired by the idea of the expected value of perfect information: a good question is one whose expected answer is going to be most useful. To build generalizable systems, we propose two future research directions: a template-based model and a sequence-to-sequence based neural generative model.
