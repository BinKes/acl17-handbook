When it comes to computational language processing systems, humour is a relatively unexplored domain, especially more so for Hindi (or rather, most languages other than English). Most researchers agree that a joke consists of two main parts - the setup and the punchline, which the humour being encoded in the incongruity between the two. In this paper, we look at Dur se Dekha jokes, a restricted domain of humorous three liner poetry in Hindi. We analyze their structure to understand how humour is encoded in them and formalize it. We then develop a system which is successfully able to generate a basic form of these jokes.
