Schizophrenia is one of the most disabling and difficult to treat of all human medical/health conditions, ranking in the top ten causes of disability worldwide. It has been a puzzle in part due to difficulty in identifying its basic, fundamental components. Several studies have shown that some manifestations of schizophrenia (e.g., the negative symptoms that include blunting of speech prosody, as well as the disorganization symptoms that lead to disordered language) can be understood from the perspective of linguistics. However, schizophrenia research has not kept pace with technologies in computational linguistics, especially in semantics and pragmatics. As such, we examine the writings of schizophrenia patients analyzing their syntax, semantics and pragmatics. In addition, we analyze tweets of (self proclaimed) schizophrenia patients who publicly discuss their diagnoses. For writing samples dataset, syntactic features are found to be the most successful in classification whereas for the less structured Twitter dataset, a combination of  features performed the best.
