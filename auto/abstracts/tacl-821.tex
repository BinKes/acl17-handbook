Algorithmic decipherment is a prime example of a truly unsupervised problem. The first step in the decipherment process is the identification of the encrypted language. We propose three methods for determining the source language of a document enciphered with a monoalphabetic substitution cipher. The best method achieves 97\% accuracy on 380 languages. We then present an approach to decoding anagrammed substitution ciphers, in which the letters within words have been arbitrarily transposed. It obtains the average decryption word accuracy of 93\% on a set of 50 ciphertexts in 5 languages. Finally, we report the results on the Voynich manuscript, an unsolved fifteenth century cipher, which suggest Hebrew as the language of the document.