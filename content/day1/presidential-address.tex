\section{Presidential Address: Joakim Nivre}
\index{Nivre, Joakim}

\begin{center}
\begin{Large}
{\bfseries\Large ACL 2017 Presidential Address}\vspace{1em}\par
\end{Large}

\daydateyear, 9:00--9:10 \vspace{1em}\\
\PlenaryLoc \\
\vspace{1em}\par
\end{center}

\noindent
{\bfseries Abstract:} Computational linguistics is a booming field and our association is flourishing with it. As our conferences grow larger and the pace of publishing quickens, there is a constant need to reflect on strategies that will allow us to prosper and grow even stronger in the future. In my presidential address, I will focus on three topics that I think require our attention. The first is equity and diversity, where the ACL executive committee has recently launched a number of actions intended to improve the inclusiveness and diversity of our community, but where there is clearly a need to do more. The second topic is publishing and reviewing, where the landscape is changing very quickly and our current system is starting to strain under the sheer volume of submissions. In particular, there has been an active discussion recently about the pros and cons of preprint publishing and the way it interacts with our standard model for double-blind reviewing. On this topic, I will present the results of a large-scale survey organized by the ACL executive committee to learn more about current practices and views in our community, a survey that will be followed up by a panel and discussion at the ACL business meeting later in the week. The third and final topic is good science and what we can do to promote scientific methodology and research ethics, which is becoming increasingly important in a world where the role of science in society cannot be taken for granted.

\vspace{3em}\par 

\vfill
\noindent

{\bfseries Biography:} Joakim Nivre is Professor of Computational Linguistics at Uppsala University. He holds a Ph.D. in General Linguistics from the University of Gothenburg and a Ph.D. in Computer Science from Växjö University. His research focuses on data-driven methods for natural language processing, in particular for syntactic and semantic analysis. He is one of the main developers of the transition-based approach to syntactic dependency parsing, described in his 2006 book Inductive Dependency Parsing and implemented in the widely used MaltParser system, and one of the founders of the Universal Dependencies project, which aims to develop cross-linguistically consistent treebank annotation for many languages and currently involves over 150 researchers around the world. He has produced over 200 scientific publications and has more than 11,000 citations according to Google Scholar (June, 2017). He is currently the president of the Association for Computational Linguistics. 

\newpage
