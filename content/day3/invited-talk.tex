\section{Keynote Address: Mirella Lapata}
\index{Lapata, Mirella}

\begin{center}
\begin{Large}
{\bfseries\Large Translating from Multiple Modalities to Text and Back}\vspace{1em}\par
\end{Large}

\daydateyear, 9:00--10:10 \vspace{1em}\\
\PlenaryLoc \\
\vspace{1em}\par
\end{center}

\noindent
{\bfseries Abstract:} Recent years have witnessed the development of a wide range of computational tools that process and generate natural language text. Many of these have become familiar to mainstream computer users in the from of web search, question answering, sentiment analysis, and notably machine translation. The accessibility of the web could be further enhanced with applications that not only translate between different languages (e.g., from English to French) but also within the same language, between different modalities, or different data formats. The web is rife with non-linguistic data (e.g., video, images, source code) that cannot be indexed or searched since most retrieval tools operate over textual data.

In this talk I will argue that in order to render electronic data more accessible to individuals and computers alike, new types of translation models need to be developed.  I will focus on three examples, text simplification, source code generation, and movie summarization. I will illustrate how recent advances in deep learning can be extended in order to induce general representations for different modalities and learn how to translate between these and natural language.

\vspace{3em}\par 

\vfill
\noindent

{\bfseries Biography:} Mirella Lapata is professor of natural language processing in the School of Informatics at the University of Edinburgh. Her research focuses on getting computers to understand, reason with, and generate. She is as an associate editor of the Journal of Artificial Intelligence Research and has served on the editorial boards of Transactions of the ACL and Computational Linguistics. She was the first recipient of the Karen Sparck Jones award of the British Computer Society, recognizing key contributions to NLP and information retrieval. She received two EMNLP best paper awards and currently holds a prestigious Consolidator Grant from the European Research Council.

\newpage
