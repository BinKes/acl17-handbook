\section{Message from the Program Committee Co-Chairs}
\setheaders%
    {Message from the Program Committee Co-Chairs}%
    {Message from the Program Committee Co-Chairs}
\thispagestyle{emptyheader}
%\renewcommand{\large}{\fontsize{9}{11}\selectfont}
% that's a hack to make this part nicely fill the pages

\setlength{\parskip}{.7ex}
%\setlength{\parindent}{0pt}

Welcome to the 55th Annual Meeting of the Association for
Computational Linguistics! This year, ACL received 751 long paper
submissions and 567 short paper submissions\footnote{These numbers
  exclude papers that were not reviewed due to formatting, anonymity,
  or double submission violations or that were withdrawn prior to
  review, which was unfortunately a substantial number.}. Of the long
papers, 195 were accepted for presentation at ACL — 117 as oral
presentations and 78 as poster presentations (25\% acceptance rate).
107 short papers were accepted — 34 as oral and 73 as poster
presentations (acceptance rate of 18\%). In addition, ACL will also
feature 21 presentations of papers accepted in the {\it Transactions
  of the Association for Computational Linguistics} (TACL). Including
the student research workshop and software demonstrations, the ACL
program swells to a massive total of 367 paper presentations on the
scientific program, representing the largest ACL program to date.

% Min: invited speakers
ACL 2017 will have two distinguished invited speakers: Noah A. Smith
(Associate Professor of Computer Science and Engineering at the
University of Washington) and Mirella Lapata (Professor in the School
of Informatics at the University of Edinburgh).  Both are
well-renowned for their contributions to the field of computational
linguistics and are excellent orators.  We are honored that they have
accepted our invitation to address the membership at this exciting
juncture in our field's history, addressing key issues in
representation learning and multimodal machine translation.

% Min: new innovations
To manage the tremendous growth of our field, we introduced some
changes to the conference. With the rotation of the annual meeting to
the Americas, we anticipated a heavy load of submissions and early on
we decided to have both the long and short paper deadlines merged to
reduce reviewing load and to force authors to take a stand on their
submissions' format.  The joint deadline allowed us to only load our
reviewers once, and also enabled us to have an extended period for
more lengthy dialogue among authors, reviewers and area chairs.  

In addition, oral presentations were shortened to fourteen (twelve)
minutes for long (short) papers, plus time for questions.  While this
places a greater demand on speakers to be concise, we believe it is
worth the effort, allowing far more work to be presented orally. We
also took advantage of the many halls available and expanded the
number of parallel talks to five during most of the conference
sessions.

In keeping with changes introduced in the ACL community from last
year, we continued the practice of recognizing outstanding papers at
ACL. The 22 outstanding papers (15 long, 7 short, 1.6\% of
submissions) represent a broad spectrum of exciting contributions and
have been specially placed on the final day of the main conference
where the program is focused into two parallel sessions of these
outstanding contributions. From these, a best paper and a best short
paper those will be announced in the awards session on Wednesday
afternoon.

%% Following other recent ACL conferences, submissions were reviewed
%% under different categories and using different review forms for
%% empirical/data-driven, theoretical, applications/tools,
%% resources/evaluation, and survey papers. We introduced special fields
%% in the paper submission form for authors to explicitly note the
%% release of open-source implementations to enable reproducibility, and
%% to note freely available datasets.  We also allowed authors to submit
%% appendices of arbitrary length for details that would enable
%% replication; reviewers were not expected to read this material.

%% Another innovation we explored during the review period was the
%% scheduling of short paper review before long paper review. While this
%% was planned to make the entire review period more compact (fitting
%% between the constraints of NAACL 2016 and EMNLP 2016 at either end),
%% we found that reviewing short papers first eliminated many of the
%% surprises for the long paper review process.

%% We sought to follow recently-evolved best practices in planning the
%% poster sessions, so that the many high-quality works presented in that
%% format will be visible and authors and attendees benefit from the
%% interactions during the two poster sessions.

Chris has already mentioned our introduction of the chairs'
blog\footnote{\url{https://chairs-blog.acl2017.org/}}, where we strove
to make the selection process of the internal workings of the
scientific committee more transparent.  We have publicly documented
our calls for area chairs, reviewers and accepted papers selection
process.  Via the blog, we communicated several innovations in the
conference organization workflow, of which we would call attention to
two key ones here.

In the review process, we pioneered the use of the Toronto Paper
Matching System, a topic model based approach to the assignment of
reviewers to papers.  We hope this decision will spur other program
chairs to adopt the system, as increased coverage will better the
reviewer/submission matching process, ultimately leading to a higher
quality program.

For posterity, we also introduced the usage of hyperlinks in the
bibliography reference sections of papers, and have worked with the
ACL Anthology to ensure that digital object identifiers (DOIs) appear
in the footer of each paper.  These steps will help broaden the
long-term impact of the work that our community has on the scientific
world at large.

There are many individuals we wish to thank for their contributions
to ACL 2017, some multiple times:

\begin{itemize}
\item The 61 area chairs who volunteered for our extra duty. They
  recruited reviewers, led discussions on each paper, replied to
  authors' direct comments to them and carefully assessed each
  submission.  Their input was instrumental in guiding the final
  decisions on papers and selecting the outstanding papers.
%
% Mausam, Omri Abend, Eugene Agichtein, Ron Artstein, Alexandra Balahur,
% Mohit Bansal, Chia-Hui Chang, Grzegorz Chrupała, Mona Diab, Jason
% Eisner, Manaal Faruqui, Raquel Fernandez, Karën Fort, Amir Globerson,
% Hannaneh Hajishirzi, Chiori Hori, Tommi Jaakkola, Yangfeng Ji, Jing
% Jiang, Sarvnaz Karimi, Anna Korhonen, Zornitsa Kozareva, Lun-Wei Ku,
% Nate Kushman, Chia-ying Lee, Oliver Lemon, Roger Levy, Sujian Li,
% Wenjie Li, Kang Liu, Tie-Yan Liu, Yang Liu, Zhiyuan Liu, Minh-Thang
% Luong, Saif M Mohammad, Alexander M Rush, Haitao Mi, Alessandro
% Moschitti, Smaranda Muresan, Preslav Nakov, Graham Neubig, Aurélie
% Névéol, Shimei Pan, Michael Piotrowski, Emily Pitler, Barbara Plank,
% Sujith Ravi, Verena Rieser, Sophie Rosset, Mehroosh Sadrzadeh, Hinrich
% Schütze, Anders Søgaard, Karin Verspoor, Aline Villavicencio, Svitlana
% Volkova, Bonnie Webber, Deyi Xiong, William Yang Wang, Wajdi
% Zaghouani, Yue Zhang and Hai Zhao.

\item Our full program committee of BUG hard-working individuals who
  reviewed the conference’s 1,318 submissions (including secondary
  reviewers).
\item TACL editors-in-chief Mark Johnson, Lillian Lee, and Kristina
  Toutanova, for coordinating with us on TACL presentations at ACL.
\item Noah Smith and Katrin Erk, program co-chairs of ACL 2016 and Ani
  Nenkova and Owen Rambow, program co-chairs of NAACL 2016, who we
  consulted several times on short order for help and advice.
\item Wei Lu and Sameer Singh, our well-organized publication chairs,
  with direction and oversight from publication chair mentor Meg
  Mitchell.  Also, Christian Federmann who helped with the local
  handbook.
\item The responsive team at Softconf led by Rich Gerber, who worked
  quickly to resolve problems and who strove to integrate the use of
  the Toronto Paper Matching System (TPMS) for our use.
\item Priscilla Rasmussen and Anoop Sarkar and the local organization
  team, especially webmaster Nitin Madnani.
\item Chris Callison-Burch, our general chair, who kept us
  coordinated with the rest of the ACL 2017 team and helped us free
  our time to concentrate on the key duty of organizing the scientific
  program.
\item Key-Sun Choi, Jing Jiang, Graham Neubig, Emily Pitler, and
  Bonnie Webber who carefully reviewed papers under consideration for
  best paper recognition.
\item Our senior correspondents for the blog, who contributed guest
  posts and advice for writing and reviewing: Waleed Ammar, Yoav
  Artzi, Tim Baldwin, Marco Baroni, Claire Cardie, Xavier Carreras,
  Hal Daumé, Kevin Duh, Chris Dyer, Marti Hearst, Mirella Lapata,
  Emily M. Bender, Aurélien Max, Kathy McKeown, Ray Mooney, Ani
  Nenkova, Joakim Nivre, Philip Resnik, and Joel Tetreault.  Without
  them, the participation of the community through the productive
  comments, and without you the readership, our blog for disseminating
  information about the decision processes would not have been
  possible and a success.
\end{itemize}

We hope that you enjoy ACL 2017 in Vancouver!

\noindent ACL 2017 program co-chairs\\
Regina Barzilay, Massachusetts Institute of Technology\\
Min-Yen Kan, National University of Singapore
