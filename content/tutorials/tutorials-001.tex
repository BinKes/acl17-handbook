\begin{tutorial}
  {Natural Language Processing for Precision Medicine}
  {tutorial-001}
  {\daydateyear, \tutorialmorningtime}
  {\TutLocA}
\end{tutorial}

We will introduce precision medicine and showcase the vast opportunities for NLP in this burgeoning field with great societal impact. We will review pressing NLP problems, state-of-the art methods, and important applications, as well as datasets, medical resources, and practical issues. The tutorial will provide an accessible overview of biomedicine, and does not presume knowledge in biology or healthcare. The ultimate goal is to reduce the entry barrier for NLP researchers to contribute to this exciting domain.

\vspace{2ex}\centerline{\rule{.5\linewidth}{.5pt}}\vspace{2ex}
\setlength{\parskip}{1ex}\setlength{\parindent}{0ex}

  {\bfseries Hoifung Poon} is a Researcher at Microsoft Research Redmond. His research interests lie in advancing machine learning and natural language processing (NLP) to help automate discovery and decision support in precision medicine. He received his Ph.D. in computer science \& engineering at the University of Washington. His past work has been recognized with Best Paper Awards from premier NLP and machine learning venues such as NAACL-09 (unsupervised morphological segmentation), EMNLP-09 (unsupervised semantic parsing), and UAI-11 (sum-product networks).
  \index{Poon, Hoifung}

  {\bfseries Chris Quirk} is a Principal Researcher at Microsoft Research Redmond. Since joining Microsoft Research in 2001, his research has focused on effective computational systems for aiding human communication, understanding, and task completion. His primary focus is in machine translation, building practical and widely-used system implementations and authoring a number of influential papers. He has also worked in paraphrase, information extraction, and most recently biological applications of natural language processing and machine learning. He has served on numerous program committees, acted Area Chair (ACL 2010, EMNLP 2012), and is currently an action editor of the TACL journal.
  \index{Quirk, Chris}

  {\bfseries Kristina Toutanova} is a Staff Research Scientist at Google Research Seattle and affiliate faculty member at the University of Washington. In 2005, she obtained her Ph.D. from the Computer Science Department at Stanford University, where she was advised by Christopher Manning. She focuses on modeling the structure of natural language using machine learning, in the areas of semantic parsing, knowledge extraction, information retrieval, and text-to-text generation. She has coauthored more than 50 publications at refereed conferences and journals, including four papers that have won awards at conferences (EMNLP, NAACL, CoNLL, ECML). She served as a program co-chair for CoNLL 2008 and ACL 2014 and is currently serving as a co-editor-in-chief of the TACL journal.
  \index{Toutanova, Kristina}

  {\bfseries Wen-tau Yih} is a Senior Researcher at Microsoft Research Redmond. His research interests include natural language processing, machine learning and information retrieval. Yih received his Ph.D. in computer science at the University of Illinois at Urbana-Champaign. His work on joint inference using integer linear programming (ILP) helped the UIUC team win the CoNLL-05 shared task on semantic role labeling, and the approach has been widely adopted in the NLP community since then. After joining MSR in 2005, he has worked on email spam filtering, keyword extraction and search \& ad relevance. His recent work focuses on continuous semantic representations using neural networks and matrix/tensor decomposition methods, with applications in lexical semantics, knowledge base embedding and question answering. Yih received the best paper award from CoNLL-2011, an outstanding paper award from ACL-2015 and has served as area chairs (HLT-NAACL-12, ACL-14, EMNLP16,17), program co-chairs (CEAS-09, CoNLL-14) and action/associated editors (TACL, JAIR) in recent years.
  \index{Yih, Wen-tau}
