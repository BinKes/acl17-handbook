\begin{bio}
  {\bfseries Louis-Philippe Morency} is an associate professor at CMU Language
  Technology Institute where he leads the Multimodal Communication and Machine
  Learning Laboratory (MultiComp Lab). 

  {\bfseries Tadas Baltrusaitis} is a a postdoctoral researcher at the Carnegie
  Mellon University, Language Technologies Institute.

\end{bio}

\begin{tutorial}
  {Multimodal Machine Learning}
  {tutorial-final-002}
  {\daydateyear, \tutorialmorningtime}
  {\TutLocB}

Multimodal machine learning is a vibrant multi-disciplinary research field
which addresses some of the original goals of artificial intelligence by
integrating and modeling multiple communicative modalities, including
linguistic, acoustic and visual messages. With the initial research on
audio-visual speech recognition and more recently with image and video
captioning projects, this research field brings some unique challenges for
multimodal researchers given the heterogeneity of the data and the contingency
often found between modalities.

This tutorial builds upon a recent course taught at Carnegie Mellon University
during the Spring 2016 semester (CMU course 11-777) and two tutorials
presented at CVPR 2016 and ICMI 2016. The present tutorial will review
fundamental concepts of machine learning and deep neural networks before
describing the five main challenges in multimodal machine learning: (1)
multimodal representation learning, (2) translation & mapping, (3) modality
alignment, (4) multimodal fusion and (5) co-learning. The tutorial will also
present state-of-the-art algorithms that were recently proposed to solve
multimodal applications such as image captioning, video descriptions and
visual question-answer. We will also discuss the current and upcoming
challenges.

\end{tutorial}