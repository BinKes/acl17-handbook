\begin{bio}
  {\bfseries Louis-Philippe Morency} (\url{https://www.cs. cmu.edu/~morency/}) is Assistant Professor in the Language Technology Institute at the Carnegie Mellon University where he leads the Multimodal Communication and Machine Learning Laboratory (MultiComp Lab). He received his Ph.D. and Master degrees from MIT Computer Science and Artificial Intelligence Laboratory. In 2008, Dr. Morency was selected as one of ”AI’s 10 to Watch” by IEEE Intelligent Systems. He has received 7 best paper awards in multiple ACMand IEEE-sponsored conferences for his work on context-based gesture recognition, multimodal probabilistic fusion and computational models of human communication dynamics. Dr. Morency was General Chair for the International Conference on Multimodal Interaction (ICMI 2012) and the NIPS 2010 workshop on Modeling Human Communication Dynamics. He was Program Chair for ICMI 2011, 2014 and 2016, as well as the Tenth International Conference on Creating, Connecting and Collaborating through Computing in January 2012.
  \index{Morency, Louis-Philippe}
 
  {\bfseries Tadas Baltrusaitis} (\url{http://www.cl.cam. ac.uk/~tb346/}) is a post-doctoral associate at the Language Technologies Institute, Carnegie Mellon University. Before this, he was a postdoctoral research at the University of Cambridge, where he also received his PhD degree in 2014. His primary research interests lie in the automatic understanding of non-verbal human behaviour, computer vision, and multimodal machine learning. His papers have won a number of awards for his work on non-verbal human behavior analysis, including ICMI 2014 best student paper award, and ETRA 2016 emerging investigator award. He is also a winner of several challenges in computer vision and multi-modal machine learning, including FERA 2015, and AVEC 2011.
  \index{Baltrusaitis, Tadas}
\end{bio}

\begin{tutorial}
  {Multimodal Machine Learning}
  {tutorial-002}
  {\daydateyear, \tutorialmorningtime}
  {\TutLocB}

Multimodal machine learning is a vibrant multi-disciplinary research field which addresses some of the original goals of artificial intelligence by integrating and modeling multiple communicative modalities, including linguistic, acoustic and visual messages. With the initial research on audio-visual speech recognition and more recently with image and video captioning projects, this research field brings some unique challenges for multimodal researchers given the heterogeneity of the data and the contingency often found between modalities.
 
This tutorial builds upon a recent course taught at Carnegie Mellon University during the Spring 2016 semester (CMU course 11-777) and two tutorials presented at CVPR 2016 and ICMI 2016. The present tutorial will review fundamental concepts of machine learning and deep neural networks before describing the five main challenges in multimodal machine learning: (1) multimodal representation learning, (2) translation \& mapping, (3) modality alignment, (4) multimodal fusion and (5) co-learning. The tutorial will also present state-of-the-art algorithms that were recently proposed to solve multimodal applications such as image captioning, video descriptions and visual question-answer. We will also discuss the current and upcoming challenges.
\end{tutorial}
