\begin{bio}
  {\bfseries Yun-Nung Chen} holds a Ph.D. degree in the Language Technologies
  Institute (LTI) of School of Computer Science (SCS) at Carnegie Mellon
  University (CMU) under supervision of Prof. Alexander I. Rudnicky and
  Prof. Anatole Gershman, earned in 2015.

  {\bfseries Asli Celikyilmaz} is a researcher at Deep Learning Team at
  Microsoft Research. In her research, she focuses on learning algorithms
  relating to understanding natural language.

  {\bfseries Dilek Hakkani-Tur} is is a research scientist at Google. Prior
  to joining Google, she was a researcher at Microsoft Research(2010-2016),
  International Computer Science Institute (ICSI, 2006-2010) and AT\&T
  Labs-Research (2001-2005). Her research interests include natural language
  and speech processing, spoken dialogue systems, and machine learning for
  language processing.

\end{bio}

\begin{tutorial}
  {Deep Learning for Dialogue Systems}
  {tutorial-final-004}
  {\daydateyear, \tutorialafternoontime}
  {\TutLocD}

In the past decade, goal-oriented spoken dialogue systems have been the most
prominent component in today's virtual personal assistants. The classic
dialogue systems have rather complex and/or modular pipelines. The advance of
deep learning technologies has recently risen the applications of neural
models to dialogue modeling. However, how to successfully apply deep learning
based approaches to a dialogue system is still challenging. Hence, this
tutorial is designed to focus on an overview of the dialogue system
development while describing most recent research for building dialogue
systems and summarizing the challenges, in order to allow researchers to
study the potential improvements of the state-of-the-art dialogue systems. The
tutorial material is available at \url{http://deepdialogue.miulab.tw}.

\end{tutorial}