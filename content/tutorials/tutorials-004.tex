
\begin{tutorial}
  {Deep Learning for Dialogue Systems}
  {tutorial-004}
  {\daydateyear, \tutorialafternoontime}
  {\TutLocD}
\end{tutorial}

With the rising trend of artificial intelligence, more and more devices have incorporated goal-oriented spoken dialogue systems. Among popular virtual personal assistants, Microsoftís Cortana, Appleís Siri, Amazon Alexa, Google Assistant, and Facebookís M, have incorporated dialogue system modules in various devices, which allow users to speak naturally in order to finish tasks more efficiently.

The traditional conversational systems have rather complex and/or modular pipelines. The advance of deep learning technologies has recently risen the applications of neural models to dialogue modeling. Nevertheless, applying deep learning technologies for building robust and scalable dialogue systems is still a challenging task and an open research area as it requires deeper understanding of the classic pipelines as well as detailed knowledge on the benchmark of the models of the prior work and the recent state-of-the-art work.

The goal of this tutorial is to provide the audience with developing trend of the dialogue systems, and a roadmap to get them started with the related work. In the first section of the tutorial, we motivate the work on conversation-based intelligent agents, in which the core underlying system is task-oriented dialogue systems. The second and third sections describe different approaches using deep learning for each component in the dialogue system and how it is evaluated. The last two sections focus on discussing the recent trends and current challenges on dialogue system technology and summarize the challenges and conclusions. The tutorial material is available at \url{http://deepdialogue.miulab.tw/}.

\vspace{2ex}\centerline{\rule{.5\linewidth}{.5pt}}\vspace{2ex}
\setlength{\parskip}{1ex}\setlength{\parindent}{0ex}

  {\bfseries Yun-Nung Chen} is currently an assistant professor at the Department of Computer Science, National Taiwan University. She earned her Ph.D. degree from Carnegie Mellon University, where her research interests focus on spoken dialogue system, language understanding, natural language processing, and multi-modal speech applications. She received the Google Faculty Research Awards 2016, two Student Best Paper Awards from IEEE SLT 2010 and IEEE ASRU 2013, a Student Best Paper Nominee from Interspeech 2012, and the Distinguished Master Thesis Award from ACLCLP. Before joining National Taiwan University, she worked in the Deep Learning Technology Center at Microsoft Research Redmond. More information about her can be found at \url{http://vivianchen.idv.tw}.
  \index{Chen, Yun-Nung}

  {\bfseries Asli Celikyilmaz} is currently a researcher at the Deep Learning Technology Center at Microsfot Research. Previously, she was a Research Scientist at Microsoft Bing from 2010 until 2016 focusing on deep learning models for scaling natural user interfaces to multiple domains. She has worked as a Postdoc Researcher at the EECS Department of the UC Berkeley from 2008 until 2010. She has worked with researchers at ICSI @ Berkeley during her postdoc research study. She earned her Ph.D. from University of Toronto, Canada in 2008. Asli’s research interests are mainly machine learning and its applications to conversational dialogue systems, mainly natural language understanding and dialogue modeling. In the past she has also focused on research areas including machine intelligence, semantic tagging of natural user utterances of human to machine conversations, text analysis, document summarization, question answering, co-reference resolution, to name a few. Currently she is focusing on reasoning, attention, memory networks as well as multi-task and transfer learning for conversational dialogue systems. She has been serving as area chair, co-organizer of numerous NLP and speech conferences, such as ACL, NAACL, Interspeech, and IEEE Spoken Language Technologies (SLT). She co-organized a NIPS workshop on Machine Learning for Spoken Language Understanding and Interactions in 2015.
  \index{Celikyilmaz, Asli}

  {\bfseries Dilek Hakkani-Tur} is a research scientist at Google Research. Prior to joining Google, she was a researcher at Microsoft Research (2010-2016), International Computer Science Institute (ICSI, 2006-2010) and AT\&T Labs-Research (20012005). She received her BSc degree from Middle East Technical Univ, in 1994, and MSc and PhD degrees from Bilkent Univ., Department of Computer Engineering, in 1996 and 2000, respectively. Her research interests include natural language and speech processing, spoken dialogue systems, and machine learning for language processing. She has over 50 patents that were granted and co-authored more than 200 papers in natural language and speech processing. She is the recipient of three best paper awards for her work on active learning for dialogue systems, from IEEE Signal Processing Society, ISCA and EURASIP. She was an associate editor of IEEE Transactions on Audio, Speech and Language Processing (20052008), member of the IEEE Speech and Language Technical Committee (2009-2014), area editor for speech and language processing for Elsevier’s Digital Signal Processing Journal and IEEE Signal Processing Letters (2011-2013), and currently serves on ISCA Advisory Council (20152018). She is a fellow of IEEE and ISCA. 
  \index{Hakkani-Tur, Dilek}
