\begin{bio}
  {\bfseries Valia Kordoni} is a professor at Humboldt University in Berlin,
  Germany. 

\end{bio}

\begin{tutorial}
  {Beyond Words: Deep Learning for Multi-word Expressions and Collocations}
  {tutorial-final-005}
  {\daydateyear, \tutorialafternoontime}
  {\TutLocE}

Deep learning has recently shown much promise for NLP applications.
Traditionally, in most NLP approaches, documents or sentences are represented
by a sparse bag-of-words representation. There is now a lot of work which goes
beyond this by adopting a distributed representation of words, by constructing
a so-called ``neural embedding'' or vector space representation of each word or
document. The aim of this tutorial is to go beyond the learning of word
vectors and present methods for learning vector representations for Multiword
Expressions and bilingual phrase pairs, all of which are useful for various NLP
applications.

This tutorial aims to provide attendees with a clear notion of the linguistic
and distributional characteristics of Multiword Expressions (MWEs), their
relevance for the intersection of deep learning and natural language
processing, what methods and resources are available to support their use, and
what more could be done in the future. Our target audience are researchers and
practitioners in machine learning, parsing (syntactic and semantic) and
language technology, not necessarily experts in MWEs, who are interested in
tasks that involve or could benefit from considering MWEs as a pervasive
phenomenon in human language and communication.

\end{tutorial}