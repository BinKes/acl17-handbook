\begin{bio}
  {\bfseries Valia Kordoni} is a research professor of computational linguistics at Humboldt University Berlin. She is a leader in EU-funded research in Machine Translation, Computational Semantics, and Machine Learning. She has organized conferences and workshops dedicated to research on MWEs, recently including the EACL 2014 10th Workshop on Multiword Expressions (MWE 2014) in Gothenburg, Sweden, the NAACL 2015 11th Workshop on Multiword Expressions in Denver, Colorado, and the ACL 2016 12th Workshop on Multiword Expressions in Berlin, Germany, among others. She has been the Local Chair of ACL 2016 The 54th Annual Meeting of the Association for Computational Linguistics which took place at the Humboldt University Berlin in August 2016. She has taught a tutorial on Robust Automated Natural Language Processing with Multiword Expressions and Collocations in ACL 2013, as well as a tutorial on Robust Semantic Analysis of Multiword Expressions with FrameNet in EMNLP 2015, together with Miriam R. L. Petruck. She is also the author of Multiword Expressions From Linguistic Analysis to Language Technology Applications (to appear, Springer).
  \index{Kordoni, Valia}
\end{bio}

\begin{tutorial}
  {Beyond Words: Deep Learning for Multi-word Expressions and Collocations}
  {tutorial-005}
  {\daydateyear, \tutorialafternoontime}
  {\TutLocE}

Deep learning has recently shown much promise for NLP applications. Traditionally, in most NLP approaches, documents or sentences are represented by a sparse bag-of-words representation. There is now a lot of work which goes beyond this by adopting a distributed representation of words, by constructing a so-called ``neural embedding'' or vector space representation of each word or document. The aim of this tutorial is to go beyond the learning of word vectors and present methods for learning vector representations for Multiword Expressions and bilingual phrase pairs, all of which are useful for various NLP applications.
 
This tutorial aims to provide attendees with a clear notion of the linguistic and distributional characteristics of Multiword Expressions (MWEs), their relevance for the intersection of deep learning and natural language processing, what methods and resources are available to support their use, and what more could be done in the future. Our target audience are researchers and practitioners in machine learning, parsing (syntactic and semantic) and language technology, not necessarily experts in MWEs, who are interested in tasks that involve or could benefit from considering MWEs as a pervasive phenomenon in human language and communication.
\end{tutorial}
