\begin{bio}
  {\bfseries Jennifer Wortman Vaughan} is a Senior Researcher at Microsoft Research, New York City, where she studies algorithmic economics, machine learning, and social computing, with a heavy focus on prediction markets and other forms of crowdsourcing. She is interested in developing general methods that allow us to reason formally about the performance of algorithms with human components in the same way that traditional computer science techniques allow us to formally reason about algorithms that run on machines alone. Jenn came to Microsoft in 2012 from UCLA, where she was an assistant professor in the computer science department. She completed her Ph.D. at the University of Pennsylvania in 2009, and subsequently spent a year as a Computing Innovation Fellow at Harvard. She is the recipient of Penn’s 2009 Rubinoff dissertation award for innovative applications of computer technology, a National Science Foundation CAREER award, a Presidential Early Career Award for Scientists and Engineers (PECASE), and a handful of best paper or best student paper awards. In her “spare” time, Jenn is involved in a variety of efforts to provide support for women in computer science; most notably, she co-founded the Annual Workshop for Women in Machine Learning, which has been held each year since 2006.
\end{bio}

\begin{tutorial}
  {Making Better Use of the Crowd}
  {tutorial-final-006}
  {\daydateyear, \tutorialafternoontime}
  {\TutLocF}

Over the last decade, crowdsourcing has been used to harness the power of human computation to solve tasks that are notoriously difficult to solve with computers alone, such as determining whether or not an image contains a tree, rating the relevance of a website, or verifying the phone number of a business.

The natural language processing community was early to embrace crowdsourcing as a tool for quickly and inexpensively obtaining annotated data to train NLP systems. Once this data is collected, it can be handed off to algorithms that learn to perform basic NLP tasks such as translation or parsing.

Usually this handoff is where interaction with the crowd ends. The crowd provides the data, but the ultimate goal is to eventually take humans out of the loop. Are there better ways to make use of the crowd?

In this tutorial, I will begin with a showcase of innovative uses of crowdsourcing that go beyond data collection and annotation. I will discuss applications to natural language processing and machine learning, hybrid intelligence or “human in the loop” AI systems that leverage the complementary strengths of humans and machines in order to achieve more than either could achieve alone, and large scale studies of human behavior online.

I will then spend the majority of the tutorial diving into recent research aimed at understanding who crowdworkers are, how they behave, and what this should teach us about best practices for interacting with the crowd.

I’ll start by debunking the common myth among researchers that crowdsourcing platforms are riddled with bad actors out to scam requesters. In particular, I’ll describe the results of a research study that showed that crowdworkers on the whole are basically honest.

I’ll talk about experiments that have explored how to boost the quality and quantity of crowdwork by appealing to both well-designed monetary incentives (such as performance-based payments) and intrinsic sources of motivation (such as piqued curiosity or a sense of doing meaningful work).

I’ll then discuss recent research—both qualitative and quantitative—that has opened up the black box of crowdsourcing to uncover that crowdworkers are not independent contractors, but rather a network with a rich communication structure.

Taken as a whole, this research has a lot to teach us about how to most effectively interact with the crowd. Throughout the tutorial I’ll discuss best practices for engaging with crowdworkers that are rarely mentioned in the literature but make a huge difference in whether or not your research studies will succeed. (Here’s a few hints: Be respectful. Be responsive. Be clear.)
\end{tutorial}
